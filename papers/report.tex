%==================================================================================================
% LaTeX paper template - use as a starting point for structuring a research paper.
%
% Written by Colin Perkins (https://csperkins.org/)
% 2002-2018
%
% To the extent possible under law, the author(s) have dedicated all copyright and
% related and neighbouring rights to this software to the public domain worldwide.
% This template is distributed without any warranty.
%
% You should have received a copy of the CC0 Public Domain Dedication along with
% this software. If not, see <http://creativecommons.org/publicdomain/zero/1.0/>.
%
% NOTE: The above Public Domain Dedication applies only to the LaTeX paper
% template distributed from https://github.com/csperkins/project-template
% in the file papers/example.tex. Unless explicitly stated, modifications 
% or additions to that template are copyright by their respective authors.
%==================================================================================================

%==================================================================================================
% General advice on technical writing:
%  - George Gopen and Judith Swan, "The Science of Scientific Writing",
%    American Scientist, Nov/Dec 1990. 
%    http://www.americanscientist.org/issues/num2/the-science-of-scientific-writing/1
%  - Stephen Pinker, "The Sense of Style: The Thinking Person's Guide to
%    Writing in the 21st Century", Penguin, Sept 2014. ISBN 0525427929.
%
% The paper writing advice in the comments is derived from talks and articles
% by Simon Peyton-Jones, Jim Kurose, Henning Schulzrinne, and Jim Bednar: 
%  - http://research.microsoft.com/~simonpj/papers/giving-a-talk/giving-a-talk.htm
%  - http://research.microsoft.com/~simonpj/papers/giving-a-talk/writing-a-paper-slides.pdf
%  - http://gaia.cs.umass.edu/kurose/talks/top_10_tips_for_writing_a_paper.ppt
%  - http://www-net.cs.umass.edu/kurose/writing/intro-style.html
%  - http://www.cs.columbia.edu/~hgs/etc/writing-style.html
%  - http://homepages.inf.ed.ac.uk/jbednar/writingtips.html
%  - http://www.gabbay.org.uk/blog/paper-writing.html
%  - http://homes.cs.washington.edu/~mernst/advice/write-technical-paper.html
%  - http://dx.doi.org/10.1371/journal.pcbi.1005619
%
% LaTeX usage notes:
%  - http://www.read.seas.harvard.edu/~kohler/latex.html
%
%==================================================================================================

%==================================================================================================
% The \documentclass{} macro specifies the overall style. For computer
% networking papers, common options are:
%
%   \documentclass[twocolumn,a4paper]{article}  % Base LaTeX style
%   \documentclass[conference]{IEEEtran}        % IEEE Conference
%   \documentclass[10pt,sigconf]{acmart}        % ACM SIGCOMM conference
%
% The IEEE and ACM templates are included in the lib/tex/inputs directory,
% but check for updates and use the version specified by the conference or
% journal to which you're submitting.

\documentclass[10pt,sigconf]{acmart}

% The following packages are recommended, and should be available in most
% standard LaTeX installations (or from https://www.ctan.org/). Note that 
% the order in which packages are loaded is significant.

% Basic extensions recommended for all LaTeX documents:
%   nag        Warn about common problems with LaTeX files
%   inputenc   Specify the character set used in .tex files
%   babel      Language-specific typography and hyphenation
%   microtype  Improved typography when generating PDF files

\usepackage[l2tabu,orthodox]{nag}
\usepackage[utf8x]{inputenc}
\usepackage[british]{babel}
\usepackage{ifpdf}
\ifpdf
  \usepackage{microtype}
\fi

% The AMS mathematics library greatly extends and improves mathematics
% support in LaTeX (see http://ctan.org/pkg/amsmath for details). When
%  preparing multi-line numbered equations, be sure to use:
%
%   \begin{align}
%     ...
%   \end{align} 
%
% rather than:
%
%   \begin{eqnarray}
%     ...
%   \end{eqnarray}.  
%
% when this package is loaded, to ensure they formatting is consistent
% (see https://tug.org/pracjourn/2006-4/madsen/madsen.pdf for details).

\usepackage{amsmath}
\usepackage[all]{onlyamsmath}

% Use Times, Helvetica, and Courier fonts, rather than Computer Modern:

\makeatletter
\@ifclassloaded{acmart}{
  % The ACM article style sets the fonts internally
}{
  \usepackage{newtxtext}
  \usepackage{newtxmath}
}
\makeatother

% Add support for sub-figures within a figure, as follows:
%
%   \begin{figure}
%     \centering
%     \subfloat[caption for 1st subfloat]{
%       \includegraphics{...}
%       \label{...}
%     }
%     \\
%     \subfloat[caption for 2nd subfloat]{
%       \includegraphics{...}
%       \label{...}
%     }
%     \caption{caption for entire figure}
%     \label{...}
%   \end{figure*}
%
% The subfig package obsoletes the older subfigure package, and is itself
% deprecated in favour of the subcaption package. However, as of April 2015
% subcaption doesn't work with ACM or IEEE style files (this is also the
% reason for the [caption=false] option).

\usepackage[caption=false]{subfig}

% Improve formatting of tables. To produce nice looking tables:
%
%   - avoid vertical lines;
%   - avoid double horizontal lines;
%   - use horizontal lines above and below the table, and to separate the
%     header from the body of the table, but not elsewhere; and
%   - if in doubt, align columns to the left (columns of numbers should
%     align to the decimal point)
%
% This translates to a tabular environment that looks something like the
% following:
%
%   \begin{tabular}{lll}
%     \toprule
%        Header1 & Header2 & Header3 \\
%     \midrule
%        Line1   & ...     & ...     \\
%        Line2   & ...     & ...     \\
%        Line3   & ...     & ...     \\
%        Line4   & ...     & ...     \\
%     \bottomrule
%   \end{tabular}

\usepackage{booktabs}

% Improve formatting for quote marks in verbatim mode:
\usepackage{upquote}

% Improve support for graphics:
\usepackage{graphicx}

% Add support for URLs using \url{...}. This formats the URL in typewriter
% font, and makes it a hyperlink if the hyperref package is also loaded.
\usepackage{url}

% Add support for drawing packet headers. For instructions, see
% http://ctan.org/tex-archive/macros/latex/contrib/bytefield
\usepackage{bytefield}

% Add support for typesetting program source code. You can either include
% code in-line:
%
%   \begin{lstlisting}[language=Python]
%   Source code goes here
%   \end{lstlisting}
%
% or include a source file:
%
%  \lstinputlisting[language=Python]{source_filename.py}
%
% This package is highly customisable and supports a range of languages.
% See package documentation at https://ctan.org/pkg/listings for details.
\usepackage{listings}

% Generated PDF files can include hyperlinks for URLs and cross-references
% using the hyperref package. This package, however, can interact poorly
% with others. Known issues include:
%
%  - Papers typeset without page numbers gives warnings of the form:
%      "pdfTeX warning (ext4): destination with the same identifier 
%      (name{page.}) has been already used, duplicate ignored".
%    since hyperref tries to refer to the page number.
%  - The algorithmic package uses the same line-numbering scheme for each
%    algorithm, and can cause duplicate identifier warnings if you have
%    several algorithms with line numbers (this may have been fixed with 
%    recent versions of algorithmic...).
%  - If using the algorithm package with hyperref, you need to load packages
%    in the following order (see README in hyperref documentation):
%      \usepackage{float}
%      \usepackage{hyperref}
%      \usepackage{algorithm}
% For these reasons, hyperref is best to avoid for most papers, however if
% needed, uncomment the following two lines:
%  \usepackage{float}
%  \usepackage{hyperref}

% The algorithm package defines the algorithm environment. This is used in
% the same way as the figure and table environments, to include algorithms
% in a paper. The algpseudocode package provides the ability to typeset the
% algorithms: http://ctan.org/tex-archive/macros/latex/contrib/algorithmicx
\usepackage{algorithm}
\usepackage{algpseudocode}
\usepackage{color}

% By default, LaTeX adds extra space after punctuation. The \frenchspacing
% command disables this. This creates tighter looking, more even, text and
% avoids inconsistencies if you forget to use '\ ' to suppress the spacing
% after in-sentence punctuation.
\frenchspacing

% Prevent hyphenation of all upper case words:
\uchyph=0

% The ACM style needs \maketitle after the abstract, but the other styles
% want it before; these macros hide the difference and are used below:
\makeatletter
\@ifclassloaded{acmart}{
  \newcommand{\maketitleSTD}{}
  \newcommand{\maketitleACM}{\maketitle}
}{
  \newcommand{\maketitleSTD}{\maketitle}
  \newcommand{\maketitleACM}{}
}
\makeatother

% Define a simple \todo{...} macro:
\newcommand{\todo}[1]{\textbf{\textcolor{red}{To do: #1}}}

%==================================================================================================
\begin{document}
% Specify the title of the document:

\title{Year 1 PhD Progress Report}

% Specify the authors of the document. Unfortunately, there's no consistent
% way to do this that works across the different document classes. 
%
% If using \documentclass{article}:
%
%   \author{
%      A. N. Other\\University of Glasgow
%   \and
%      Colin Perkins\\University of Glasgow
%   }
%
% If using \documentclass{IEEEtran}:
%
%   \author{
%     \IEEEauthorblockN{A. N. Other}
%     \IEEEauthorblockA{University of Glasgow}
%   \and
%     \IEEEauthorblockN{Colin Perkins}
%     \IEEEauthorblockA{University of Glasgow}
%   }
%
% If using \documentclass{acmart} add a block like the following per author:
%
%   \author{Colin Perkins}
%   \orcid{0000-0002-3404-8964}
%   \affiliation{
%     \institution{University of Glasgow}
%     \streetaddress{School of Computing Science}
%     \city{Glasgow}
%     \postcode{G12 8QQ}
%     \country{UK}
%   }
%   \email{csp@csperkins.org}
%
% If you don't have an ORCID identifier, sign up for one at https://orcid.org

\author{Vivian Band}
\affiliation{
  \institution{University of Glasgow}
  \streetaddress{School of Computing Science}
  \city{Glasgow}
  \postcode{G12 8QQ}
  \country{UK}
}
\email{v.band.1@research.gla.ac.uk}

% Specify metadata about the paper. Again, what is required depends on the
% document class. If using \documentclass{acmart}, specify the following:
%
%   \acmYear{2018}
%   \copyrightyear{2018}
%   \setcopyright{acmcopyright}
%   \acmConference{CoNEXT '18}{December 4--7, 2018}{Heraklion/Crete, Greece}
%   \acmPrice{15.00}
%   \acmDOI{10.1145/3284850.3284856}
%   \acmISBN{978-1-4503-6082-1/18/12}
%
% The complete metadata is likely only available when preparing the final,
% camera ready, version of the paper.

%==================================================================================================
\maketitleSTD
\begin{abstract}
  % Four sentences:
  %  - State the problem
  %  - Say why it's an interesting problem
  %  - Say what your solution achieves
  %  - Say what follows from your solution

  The abstract goes here.

\end{abstract}
\maketitleACM
%==================================================================================================
\section{Introduction}

% A good paper introduction is fairly formulaic. If you follow a simple set
% of rules, you can write a very good introduction. The following outline can
% be varied. For example, you can use two paragraphs instead of one, or you
% can place more emphasis on one aspect of the intro than another. But in all
% cases, all of the points below need to be covered in an introduction, and
% in most papers, you don't need to cover anything more in an introduction.
%
% Paragraph 1: Motivation. At a high level, what is the problem area you
% are working in and why is it important? It is important to set the larger
% context here. Why is the problem of interest and importance to the larger
% community?



% Paragraph 2: What is the specific problem considered in this paper? This
% paragraph narrows down the topic area of the paper. In the first
% paragraph you have established general context and importance. Here you
% establish specific context and background.



% Paragraph 3: "In this paper, we show that...". This is the key paragraph
% in the introduction - you summarize, in one paragraph, what are the main
% contributions of your paper, given the context established in paragraphs 
% 1 and 2. What's the general approach taken? Why are the specific results
% significant? The story is not what you did, but rather:
%  - what you show, new ideas, new insights
%  - why interesting, important?
% State your contributions: these drive the entire paper.  Contributions
% should be refutable claims, not vague generic statements.

In this paper, we ...

% Paragraph 4: What are the differences between your work, and what others
% have done? Keep this at a high level, as you can refer to future sections
% where specific details and differences will be given, but it is important
% for the reader to know what is new about this work compared to other work
% in the area.



% Paragraph 5: "We structure the remainder of this paper as follows." Give
% the reader a road-map for the rest of the paper. Try to avoid redundant
% phrasing, "In Section 2, In section 3, ..., In Section 4, ... ", etc.

We structure the remainder of this paper as follows.

%==================================================================================================
% Concentrate single-mindedly on a narrative that:
%  - Describes the problem, and why it's interesting
%  - Describes your idea
%  - Defends your idea, showing how it solves the problem, and filling out
%    the details
% On the way, cite relevant work in passing, but defer discussion to the
% end.
%
% Introduce the problem, and your idea, using examples, and only then
% present the general case. Explain the idea as if your were speaking to
% someone using a whiteboard. Conveying the intuition is primary; details
% follow. Write in a top down manner: state broad themes and ideas first,
% then go into details.
%
% The introduction makes claims. The body of the paper provides evidence
% to support each claim. Check each claim in the introduction, identify
% the evidence, and forward-reference it from the claim. 
%==================================================================================================
\section{Survey}

% Literature survey
Papers to include:



%==================================================================================================
\section{Description of Current Work}

% Description of what's been done this year in terms of experiments/data

A good foundation for the rest of the PhD is proving that it is actually possible for attackers to exploit exposed information in IPv6 addresses.
An IPv6 address scanner does not need to scan the entire address space in order to be a plausible attack vector, it only needs to iterate through enough valid address ranges to yield a significant number of active targets in a feasible amount of time.
It also should to be possible to construct the scanner from publicly available information; 
attackers may be able to access valid IPv6 addresses illictly through other means, but would likely opt for methods which keep them as anonymous as possible as well as granting them a unique dataset for an economic advantage over similar groups.


IPv6 addresses are 128 bits in length, and are usually formed of two 64-bit halves: the network prefix, and the interface identifier.
A common behaviour in malware variants which aim to recruit as many hosts as possible for botnets, or as space to rent out to other malware operators, is to infect an initial host through a malicious downloaded file and then scan locally to find other machines on the same network.
This approach has two advantages: circumventing firewalls, and avoiding problems with network address translation common in IPv4 addressing.
The IPv6 address space is large enough that network address translation would not be required, and attackers would not need to take this into account, but active, correctly configured firewalls would still need to be circumvented in IPv6-only networks;
initial entry into a network through infected email attachments or downloads, followed by lateral movement through the local network is therefore likely to remain a common behaviour.

\subsection{Local Host Scanning}
Lateral movement over local networks was the first scenario considered in the construction of an IPv6 scanner.
The first machine is infected with malware as a result of a user opening a malicious attachment or running a suspect file;
being able to inspect this first host gives the malicious application a wealth of information, including the 64-bit network prefix used in IPv6 addresses in this particular network.
This halves the search space, leaving the remaining 64 bits which make up the interface identifier.

The interface identifier half of IPv6 addresses can be allocated in several ways.
Stateless Address Autoconfiguration (SLAAC) allocates IPv6 addresses based on the local network prefix and the device's link-layer address (ie. the MAC address of the network interface card seeking an address).
According to RFC7707, SLAAC support is mandatory for IPv6 addressing;
the vast majority of modern operating systems support the allocation of privacy addresses, which attempt to improve security by obscuring the MAC address information in IPv6 addresses.
A future point of investigation is to find out if these privacy addresses replace the standard SLAAC address, or if these addresses simple obscure MAC addresses to passive observers;
it may well be the case that devices with privacy addresses can still be reached through their `hidden' SLAAC address in active scanning attacks.

In either case, the embedded systems used in IoT devices are unlikely to be complex enough to support the assignment of secure IPv6 privacy addresses:
they will either be too lightweight to support them at all, or they will struggle to implement a random number generator secure enough that its output cannot be predicted.
IoT devices are currently popular targets for malware campaigns given that they have a tendency to be vulnerable to exploits and are often left with default credentials.
Migration from IPv4 to IPv6 will not remedy these issues, so this initial experiment assumes that the main targets will be IoT devices with SLAAC-assigned IPv6 addresses which contain exposed MAC address information.

The first half of a MAC address (ie. the first 24 bits) is known as an organisationally unique identifier (OUI), which is specific to the manufacturer of a given network interface card;
the second half of the address is specific to that card.
In SLAAC-assigned IPv6 addresses, this OUI is used as the first 24 bits in the network interface identifier half of the address.
The constant \texttt{0xFFFE} is used in the following 16 bits of the IPv6 address.
Given this information, we can state that the final search space for vulnerable IoT devices with SLAAC-assigned IPv6 addresses on a local network is n\*2\^24, where n is the number of OUIs being searched.
As stated previously, the IPv4 address space, with 2\^32 possible addresses, can be scanned in 5 minutes;
the address space for 256 different OUIs can be scanned in this time on a local network, making this a viable attack vector for finding a wide variety of IoT devices.

\subsection{Internet-Wide Host Scanning}
Malicious actors are keen to recruit hosts as widely as possible, particularly botnet operators and outfits who lease hosts to other malware operators;
a larger number of located targets translates into a business advantage for such parties, so an automated method of boosting these numbers as much as possible is an attractive prospect.
It is therefore important to consider the use-case of attackers performing Internet-wide scans of the IPv6 address space to find targets not adequately protected by firewalls.

The Border Gateway Protocol (BGP) is used to exchange routing and reachability information between autonomous systems on the wider Internet.
BGPStream is an application which can be used to process historic BGP data gathered by points known as collectors.
Announcement messages are of particular interest to this scanning project:
these indicate that a new IPv6 network prefix was being advertised by an autonomous system, and shows an IPv6 network prefix which was in use at the time.
Each prefix is accompanied by a mask, which indicates how many bits must be set to specific values for that address;
as expected from the word `prefix', these bits are always set contiguously from the start of the address.
This is an integer usually between 1 and 64.

These announcement messages can be scraped using the BGPStream API to give a list of prefixes which were active on a given date and time.
Further filtering can be done on this list to remove any duplicate prefixes (ie. prefixes which were announced more than once), and to identify which prefixes are subsets of other prefixes
(ie. if two prefixes have the same address but different masks, the advertised prefix with the larger number as its mask is a subset of the prefix which has the smaller number, due to having more bits set to pre-determined values).

The tutorial for prefix monitoring on the BGPStream website scrapes BGP data obtained over 5 hours on August 12th 2014.
This dataset was relatively small, and therefore a useful starting point for testing the scripts which removed duplicate address announcements and counted the number of instances of each prefix mask.
This process was repeated for gathering and processing IPv6 prefix announcement data over 5 hours on May 26th 2020, by the same collector which, theoretically, had the same view of the Internet as it did in 2014;
the 2020 dataset was much larger, containing 88,139 unique advertised prefixes, compared to the 2014 set which only had 1150.

TODO: add table showing full stats for each dataset (raw numbers, totals, percentages)

Although the raw number of prefixes have markedly increased, /48 prefixes remain the most frequently announced in both datasets:
43.8\% (504 announcements) in the 2014 set and 49.0\% (43205 announcements) in the 2020 set.
This means that 48 bits of the network prefix are already known.
Combined with the earlier knowledge that only 24 bits are unknown in the interface identifier half of an IPv6 address for a device with a SLAAC-assigned address, this gives a total search space of 40 bits for prefix announcements with a /48 mask.
An attacker could iterate through all possible combinations of these 40 bits, however, they do not have to search the space exhaustively;
any network prefix which consistently returns request timeouts is likely to be firewalled or not in use.
These prefixes can be discounted from the search, and the scanner can move onto searching the next possible prefix.
As such, it may be more helpful to think of the network prefix search space as somewhat independent of the interface identifier search space:
if, after a small number of ping attempts to different possible hosts (eg.100) on the same IPv6 network, the scanner does not receive a reply or a \texttt{Destination host unreachable} response, the prefix can be assumed to be unproductive and not worth searching further.
This means that /48 prefixes effectively have a search space of 16 bits, in order to iterate through all full 64-bit IPv6 network prefixes;
a collection of active prefixes can be found in under a minute, assuming a relatively small number of attempts are made to ping different hosts on all possible iterations of these 16 bits.
Any `live' prefixes which were found during this search can then be fully scanned for the remaining unknown 24 bits in the interface identifier.

/32 announcements were the second most common announced, making up 27.8\% of the 2014 set and 16.3\% of the 2020 set.
Announced prefixes with a mask smaller than 32 bits were rare:
3.6\% in 2014 and 4.7\% in 2020.
Given that 32-bit searches are possible in around 5 minutes, and that the network prefix and interface identifier search spaces are largely independent of each other, there are a very wide range of IPv6 network prefix ranges which can be searched in a feasible amount of time:
a single /32 advertisement could be searched exhaustively for valid network prefixes in around 5 minutes if only attempting one possible interface identifier, but would take n\*5 minutes when discarding inactive prefix configurations after n attempts, as described previously.
This is still feasible, but not as likely to produce results as quickly as searching through /48 advertisements.
/48 advertisements are therefore likely to remain a more popular target for these kinds of attacks.


Fully-formed 64-bit IPv6 address prefixes can also be obtained through side-channel attacks, such as harvesting addresses from responses to phishing emails, or .
These are important security risks to consider, particularly because attacks using these approaches are less complex to perform than using BGP data as described in this section.
For obvious ethical reasons, in-depth analysis of these side-channel attacks are beyond the scope of this project, but proposed fixes to improve IPv6 security could still help to mitigate any potential damage they may cause;
improvements in IPv6 addressing security provides additional protection no matter the method of attack.

\subsection{Relevance of Work to Recent Events}
The threat posed by mass recruitment of poorly secured IoT devices is not theoretical.
Malware families like Mirai, IoT Reaper, and Mozi have taken advantage of lax IoT security in recent years to recruit hosts for botnets, with Mirai in particular amassing enough bots to generate over 1Tb/s of traffic used for distributed denial of service (DDoS) attacks.
When working as intended, Mirai denial of service attacks managed to take down Dyn, a popular DNS provider, which prevented many users from accessing popular sites like Amazon and Twitter.
When not working as intended, Mirai still managed to cause a widespread service outage for TalkTalk and Deutche Telekom customers by taking 900,000 infected routers offline;
this was caused by a buggy implementation of Mirai which unintentionally deactivated routers instead of using them to contribute to DDoS attacks.

As shown by the notable increase in the number of IPv6 prefix advertisements in the 2020 BGP dataset compared to the 2014 equivalent, IPv6 deployment is on the rise.
The number of IoT devices on the market is also increasing, with devices like Amazon Alexa being specifically marketed for domestic use by users who are unlikely to know how to secure their home networks.
There is a persistent myth that IPv6 is more resilient against scanning attacks (and therefore mass host recruitment) because the address space is too large to search exhaustively;
the aim of this work is to disprove this, and to emphasise that security through obscurity is not a reliable defense strategy when working with IPv6.
This is particularly true of IoT device manufacturers, who are infamous for treating security as an afterthought instead of a primary consideration.
Mass manufacture of devices which are vulnerable to these attacks is unethical, particularly when they are marketed at `non-technical' users as a quality-of-life improvement or when they are used for sensitive data, such as domestic video capture;
warning manufacturers about the risks of these attacks and advising on how to prevent them is important.

%==================================================================================================
\section{Thesis Statement}

IPv6 addressing is said to be resistant to scanning attacks due to exhaustive searches not being possible in a 128-bit space.
This is partially true:
exhaustive searches are not possible, but the search space can be reduced to feasible ranges through the use of publicly accessible data, such as BGP advertisements and organisationally unique identifiers.
I will prove this by proving that scanning an address space for specific classes of devices is possible, and identifying how these scanning attacks can be prevented, particularly for IoT devices which are liable to use outdated IPv6 addressing techniques.

% One or two sentences: what is the question I want to answer in this PhD?
%
% The deployment of IPv6 addressing is becoming increasingly important as the number of Internet-connected devices continues to rise rapidly, particularly the number of Internet-connected IoT devices. These IoT devices tend to be poorly secured, making them appealing targets for recruitment in botnet campaigns such as Mirai, IoT Reaper, and Mozi. Malware samples commonly find new targets by scanning the 32-bit IPv4 address space on the Internet or locally on networks infected by other means, but none are currently known to use IPv6 address space scanning for host recruitment; this is non-trivial, given that IPv6 addresses are 128-bits in length, however, there is metadata embedded in standard IPv6 addresses which attackers may be able to use to their advantage.


% Malware samples commonly perform scans of the 32-bit IPv4 address space on the Internet or locally on networks initially infected by other means to find new targets; an exhaustive scan of this space can be completed in around 5 minutes, or over several hours in order to avoid drawing attention. No samples are currently known to perform IPv6 address scans for host recruitment due to the complexity of searching a 128-bit space, however, a combination of publicly accessible network data and exposed metadata can reduce this search space down to a feasible range. This exposed metadata may in fact make IPv6 scanning more appealing to malware authors than IPv4 scans: it allows more targeted scanning which can locate specific types of device, while minimising traces of malicious scans by avoiding contact with machines this particular sample is not designed to exploit.

% The rapid increase in the number of Internet-connected devices, particularly IoT devices, necessitates increased deployment of IPv6 addresses, to restore the principle of end-to-end connectivity originally envisioned for the Internet at its inception. However, IoT devices are often poorly secured, and often either openly expose device metadata in their IPv6 addresses or rely on weak random number generation which does not adequately obscure it. When combined with publicly available network data, this metadata could be used by attackers to both reduce the IPv6 address search space to a feasible range, and to minimise traces of malicious traffic by only scanning for relevant types of devices. No malware is currently known to use IPv6 scans for host recruitment, but the prospect of stealthier scans is an incentive for malware authors to make the switch from IPv4 scanning. 

%0=================================================================================================
\section{Research Plan}


%==================================================================================================
\section{Conclusions}


%==================================================================================================
% Set the bibliography style. Choose one of the following, depending on the
% document class being used:
%
%   \bibliographystyle{abbrv}                 When using article class
%   \bibliographystyle{IEEEtran}              When using IEEE style
%   \bibliographystyle{ACM-Reference-Format}  When using ACM style

\bibliographystyle{ACM-Reference-Format}

% Load the bibliography file(s) for this paper:
\bibliography{example}

%==================================================================================================
% The following information gets written into the PDF file information:
\ifpdf
  \pdfinfo{
    /Title        (...)
    /Author       (...)
    /Subject      (...)
    /Keywords     (..., ..., ...)
    /CreationDate (D:20150827110616Z)
    /ModDate      (D:20150827110616Z)
    /Creator      (LaTeX)
    /Producer     (pdfTeX)
  }
  % Suppress unnecessary metadata, to ensure the PDF generated by pdflatex is
  % identical each time it is built. This needs pdfTeX 3.14159265-2.6-1.40.17
  % or later.
  \ifdefined\pdftrailerid
    \pdftrailerid{}
    \pdfsuppressptexinfo=15
  \fi
\fi
%==================================================================================================
\end{document}
% vim: set ts=2 sw=2 tw=75 et ai:
